For this project, we'll be modelling data generated by one of the most prevalent twitter users. The times of tweets were gathered over the course of 6 months, and stored in a file. The goal is to find a tight-fitting statistical model for it, without using an excess of parameters.

%Put pictures of data here

\section{Fitting a Non-Homogeneous Poisson Process}

The simplest place to start is with a non-homogeneous poisson process. We assume that the tweeter emits tweets according to a variable rate poisson process, and attempt to find its rate function. For convenience, we'll describe our time in hours, though naturally the units could be arbitrary. We start by simulating a poisson process of known rate, and seeing how well we can recover the original rate function. We start with a simple step function,

$$
\lambda(t) = 
\begin{cases}
5  \mbox{for} 0  \leqslant t < 30\\
10 \mbox{for} 30 \leqslant t < 50\\
5  \mbox{for} 50 \leqslant t < 100
\end{cases}
$$

And we simulate the process for 100 hours, producing a trace such as;

%Put trace here

We can fit a step function by taking the differences in emission times, then attempting to cluster them. The results from this seem to fit the rate function fairly well

%Diagram here

But if we try something more complex, eg $\lambda(t) = 5+5sin(t)$, we have some issues. This is a very high-rate process which generates a lot of data with each trace. With 100 traces, giving us roughly 3,000 data points, our clustering starts to fail after just 4 rates.

We could instead attempt to fit a polynomial function, but a smooth function is subject to one very large problem with twitter data. Humans don't behave in a smooth manner. Our usage of twitter throughout a day does not vary along some continuum, and has some odd properties. See the following trace of a single day;

% Trace of one day, showing bursts

We can see an important feature between 
%time1
and
%time2
. We'll call this a ``burst". The user is tweeting very rapidly for a short period of time. These bursts occur at random times throughout the day, and last for random lengths of time, but they all take on the same form. This heavily restricts our function. We can't simply say that at some particular time there is always a burst, but we also cannot deny their existence by smoothing them out since these bursts, by their very nature, will account for the majority of the observed tweets.

Clearly, a different approach is needed.

\section{Fitting a Markov-Modulated Poisson Process}

The Markov-Modulated Poisson Process (MMPP) is a particularly devious form of Hidden Markov Model. We assume that, underlying the tweeter, there is a CTMP. Each state is tied to a fixed Poisson Process rate. Whilst the process is in a particular state, it emits observable emissions according to a poisson process of that rate. We can simulate one of these with the following algorithm;

%MMPP Simulation algorithm goes here

Such a model seems ideal, capturing the simplicity of a step function, and letting us model the idea of randomly distributed bursts throughout a day. Indeed, several authors have postulated that such a model would be ideal for simulating human behavior %lots of citations go here
but there have so far been no actual quantifiable studies of its relevance. Part of the reason for this is likely the lack of algorithms.

Fitting a Hiddin Markov Model of any kind relies on two main algorithms, Baum-Welch %cite
and Viterbi
%cite
. The Baum-Welch Algorithm is an expectation-maximisation algorithm for finding the transition probabilities/rates and the emission probabilities, given a set of possible emissions, a number of states to fit and an observed sequence of emissions. Viterbi will take an observed sequence of emissions and the parameters of an HMM, and produce the most likely state in which each of these emissions happened. The HiddenMarkov package hosted on CRAN %cite
is the only easily-accessible Hidden Markov Model package which supports the Markov Modulated Poisson Process, but does not contain any implementation of the Viterbi algorithm. It therefore became necessary to write one.

\subsection{A derivation of the classical viterbi algorithm}

The Viterbi algorithm for a standard Discrete Time Hidden Markov Model relies on a known, finite observation space $O$, a known distribution on $O$ for each state $s$, $p_s$, and known transition probabilities $\pi_i,j$, with known initial probabilities for each state $s$, $\delta_s$. We let $[k] = \{1,2,...,k\}$. The goal is, given a sequence of T observations $(y_n)_{n \in [T]}$, $y_i \in O$, to find the sequence of states $(x_n)_{n \in [T]}$ satisfying

$$
\mathbf{x} = \argmax_{\widehat{\mathbf{x}} \in S^T} \pr(\mathbf{\widehat{x}} | \mathbf{y})
$$

We call $\mathbf{x}$ the Viterbi Path.

We let $V_{t,s}$ be the probability of the most probable state sequence responsible for the first $t$ observations which ends in state $s$ %cite http://www.cs.cmu.edu/~epxing/Class/10701/Lecture/lecture12-HMM.pdf
. We have that $V_{1,s}$ is the probability of both being in state $s$ at time 1 and seeing observation $y_1$ from state $s$. A simple application of Bayes' Rule gives us that

$$
V_{1,s} = \pr (y_1|s)\pr(x_1=s)
$$

Recall that $\delta_s$ is the probability of being in state $s$ at time 1, ie $\pr(x_1=s)$, and that $p_s(y_1)$ is the probability of observing $y_1$ from state $s$, ie $\pr (y_1|s)$. Both of these parameters are estimated by the Baum-Welch algorithm, hence,

$$
V_{1,s} = p_s(y_1)\delta_s
$$

Given $V_{\tau,s}$ for $\tau < t$, we can find $V_{t,s}$ by noting that the Markov Property implies a form of memorylessness. Given the present, the future is conditionally independent of the past. As such, we need only consider $V_{t-1,s}$ for each $s$, as well as our known parameters. 

The probability of the most likely path that leads us to state $s$ at time $t$ is given by the probability of the most likely path that led us to $s'$ at time $t-1$, and then jumped to $s$ at time $t$, and then emitted $y_t$ from state $s$. The probability of jumping from $s'$ to $s$ is $\pi_{s',s}$. The probability of emitting $y_t$ from state $s$ is $p_s(y_t)$. The probability of the most likely path that leads us to $s'$ at time $t-1$ is $V_{t-1,s'}$. Hence,

$$
V_{t,s} = p_s(y_t) \max_{s'\in S} (\pi_{s',s}V_{t-1,s'})
$$

Using this recurrence, we can find $V_{t,s} \forall t \in [T]$ by a standard dynamic programming algorithm.

From here, we can then work backwards to find the Viterbi path. $x_T = \argmax_{s \in S} V_{s,t}$ - the most likely final state is the state in which the path of maximum probability ends.

Let
 
$$
T_{s,t} = \argmax_{s'\in S}(\pi_{s',s}V_{t-1,s'})
$$

Ie, $T_{s,t}$ is the state from which we are most likely to have come at time $t-1$ given that we are in state $s$ at time $t$, we can then see that $x_{t-1} = T{x_t,t}$. Note the similarities in the definitions of $V$ and $T$ - both can be calculated simultaneously - $V$ is the maximum, $T$ is the argument that maximises. $T_{s,1}$ is never used, so need never be defined. 

Since we have an expression for $x_{t-1}$ in terms of $x_t$, and for $x_T$, we can then recover $\mathbf{x}$, the Viterbi Path. We can therefore define the viterbi algorithm as follows:

%Put algorithm here

\subsection{A Modification of the Viterbi Algorithm for the Markov-Modulated Poisson Process}

Recall the dependencies for the DTHMM Viterbi Algorithm. We require knowledge of a finite $O$, $p_s$ for each $s$, and $\pi_{ij}$ for each pair of states $i,j$.

In an MMPP, we observe a Poisson Process of rate randomly varying between various known rates - the rates being our states. Let $S = \{\lambda_1,...,\lambda_m\}$. Our observations can be interpreted as exponential random variables of these rates. Let $\tau_0 = 0$, and $\tau_i$ be the time of the $i^{th}$ poisson emission for $i \in [n]$. Let $y_i = \tau_i-\tau_{i-1}$ for $i \in [n]$. Given that the underlying CTMP was in state $\lambda_s$ at time $\tau_i$, we have that $y_i \sim Exp (\lambda_s)$. From the properties of the generic CTMP, the probability that the process is in state $j$ at time $\tau_i$ given that it was in state $i$ at time $\tau_{i-1}$ is given by $(e^{Qy_i})_{\lambda_{i},\lambda_{j}}$. To ease notation, we'll write this as $(e^{Qy_i})_{i,j}$, omitting the $\lambda$s.

The state space $S$ and transition rates $Q$ are estimated by the Baum Welch algorithm as before, so after running Baum Welch over an observed trace, we can start to find the most likey state at each emission. Note that this ``Viterbi Path" is not the most likely sequence of state transitions, it is instead the most likely state in which the underlying CTMP resides at the time of each emission. This quirk is discussed in %APPENDIX REF

Since our emissions are continuous, we don't have any notion of ``most probable" - if we model height continuously, the probability that I meet someone exactly 1.8m tall is the same as the probability that I meet someone exactly 18m tall, they're both 0 - so instead we'll base our likelihood calculations off probability density - capturing the idea that, even though I don't know for certain that I'll meet one of the two, it's more likely for me to meet the 1.8m tall person.

% NOPENOPENOPENOPENOPENOPENOPENOPENOPENOPENOPENOPENOPENOPENOPE

We let $p_s(t)= \lambda_s e^{-t\lambda_s}$, the probability density of an exponential random varaible of rate $\lambda_s$ evaluated at $t$. Let $V_{t,s}$ be the probability density of the most likely path that leads us to emitting $y_t$ from state $s$. We have that

$$
V_{1,s} =  \delta_{s}p_s(y_1)
$$

The probability density of the most likely path that leads us to waiting for time $y_1$ before making an emission is given by the probability of starting in state $s$, multiplied by the probability density of waiting $y_1$ for an emission from state $s$. The memorylessness property of a CTMP allows us to only consider $V_{t-1,s}$ when calculating $V_{t,s}$. We have that

$$
V_{t,s} = p_s(y_t) \max_{s'\in S} (V_{t-1,s'}(e^{Qy_t})_{s',s})
$$

The probability density of the most likely path that leads us to waiting for time $y_t$ between the $(t-1)^{th}$ and $t^{th}$ emissions in state $s$ is given by the probability density of the most likely path that takes us to state $s'$ for the $(t-1)^{th}$ emission, followed by jumping (along any arbitrary path) into state $s$ for emission $t$, multiplied by the probability density of emitting $y_t$ in state $s$.

From here, we can proceed as normal. We define $T$ as before to record our most likely states at each transition, and work backwards to find $\mathbf{x}$.

Interesting to note here is the fact that the algorithm assumes that either a jump happens instantaneously, or not at all - we always evaluate $p_s(y_t)$, rather than $p_s(y_t-\varepsilon)$, where $\varepsilon$ represents the time we wait for all the relevant transitions to occur. However, the times between state transitions are exponentially distributed, and we are dealing with the most likely outcomes. The most likely outcome of any exponential distribution is $0$ so, if the underlying CTMP jumps from one state to another, the most likely time for that jump to happen is immediately, so $\varepsilon = 0$

\section{The Markov-Modulated Generalised Time Process}