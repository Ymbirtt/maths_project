For this project, we'll be modelling data generated by one of the most prevalent twitter users. The times of tweets were gathered over the course of 6 months, and stored in a file. The goal is to find a tight-fitting statistical model for it, without using an excess of parameters.

%Put pictures of data here

\section{Fitting a Non-Homogeneous Poisson Process}

The simplest place to start is with a non-homogeneous poisson process. We assume that the tweeter emits tweets according to a variable rate poisson process, and attempt to find its rate function. For convenience, we'll describe our time in hours, though naturally the units could be arbitrary. We start by simulating a poisson process of known rate, and seeing how well we can recover the original rate function. We start with a simple step function,

$$
\lambda(t) = 
\begin{cases}
5  \mbox{for} 0  \leqslant t < 30
10 \mbox{for} 30 \leqslant t < 50
5  \mbox{for} 50 \leqslant t < 100
\end{cases}
$$

And we simulate the process for 100 hours, producing a trace such as;

%Put trace here

We can fit a step function by taking the differences in emission times, then attempting to cluster them. The results from this seem to fit the rate function fairly well

%Diagram here

But if we try something more complex, eg $\lambda(t) = 5+5sin(t)$, we have some issues. This is a very high-rate process which generates a lot of data with each trace. With 100 traces, giving us roughly 3,000 data points, our clustering starts to fail after just 4 rates. We could instead attempt to fit a polynomial function, but a smooth is subject to one very large problem with twitter data.

Humans don't behave in a smooth manner. Our usage of twitter throughout a day does not vary along some continuum, and has some odd properties. See the following trace of a single day;

% Trace of one day, showing bursts

We can see an important feature between 
%time1
and
%time2
. We'll call this a ``burst". The user is tweeting very rapidly for a short period of time. These bursts occur at random times throughout the day, and last for random lengths of time, but they all take on the same form. This heavily restricts our function. We can't simply say that at some particular time there is always a burst, but we also cannot deny their existence by smoothing them out since these bursts, by their very nature, will account for the majority of the observed tweets.

Clearly, a different approach is needed.

\section{Fitting a Markov-Modulated Poisson Process}

The Markov-Modulated Poisson Process (MMPP) is a particularly devious form of Hidden Markov Model. We assume that, underlying the tweeter, there is a CTMP. Each state is tied to a fixed Poisson Process rate. Whilst the process is in a particular state, it emits observable emissions according to a poisson process of that rate. We can simulate one of these with the following algorithm;

%MMPP Simulation algorithm goes here

Several people have postulated that such a model would be ideal for simulating human behavior %lots of citations go here
but there have so far been no actual studies of its relevance. Part of the reason for this is due to the lack of algorithms.

Fitting a Hiddin Markov Model of any kind relies on two main algorithms, Baum-Welch %cite
and Viterbi
%cite
. The Baum-Welch Algorithm is an expenctation-maximisation algorithm for finding the transition probabilities/rates and the emission probabilities, given a set of possible emissions, a number of states to fit and an observed sequence of emissions. Viterbi will take an observed sequence of emissions and the parameters of an HMM, and produce the most likely state in which each of these emissions happened.

\section{The Markov-Modulated Generalised Time Process}