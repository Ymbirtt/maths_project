We conclude, then, in a swarm of negative results, but with an implementation for an as yet unwritten algorithm, and exactly one statistically valid result for an as yet unsolved problem. Further refinements can, of course, be made. Various path finding algorithms can probably be adapted for a CTMP to give the most likely sequence of states between two known endpoints, which would allow us to find the modal transition time between states and create a more accurate MMPP Viterbi. We could even try for a Markov Modulated Renewal Process, the merest definition of which is non-trivial when we consider emissions happening between state transitions.

I hope that this project has shone a light into what was once a dark area, and that the resulting methods prove useful to others for detecting botnets in networks, suspicious social network activity, or simply creating some rather beautiful diagrams. As for the Markov Modulated Poisson Process, it seems that twitter already has a fitting image for the occasion, displayed on every twitter outage \cite{fail_whale}.

All code used for this project, as well as the code used to generate this document and svgs of most of the graphics are hosted on GitHub at \url{https://github.com/Ymbirtt/maths_project}. If you, for instance, had difficulty seeing some of the diagrams, or want more concrete information on exactly what happened in this project, I'd recommend finding it all there.